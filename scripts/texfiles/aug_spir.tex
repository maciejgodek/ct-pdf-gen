\documentclass[10pt, twocolumn, oneside, a4paper]{memoir}
\usepackage{fontspec}
\setmainfont{Alegreya}
\usepackage{ragged2e}
\newfontfamily\scshape{Alegreya SC}
\newfontfamily\headingfont{Alegreya Sans SC}
\newfontfamily\subheadingfont[
    BoldFont={Alegreya Sans Medium},
    ItalicFont={Alegreya Sans Light Italic},
    BoldItalicFont={Alegreya Sans Medium Italic}
]{Alegreya Sans Light}
\newfontfamily\greekfont[Script=Greek, Scale=1.14, WordSpace=0.7]{GFS Neohellenic}
\usepackage{polyglossia}
\setdefaultlanguage{english}
\setotherlanguage[variant=ancient]{greek}
\newcommand{\greektext}[1]{\foreignlanguage{greek}{#1}}

\usepackage{microtype}
\emergencystretch \hsize
\tolerance 300%
\setlength{\parskip}{0pt}
\frenchspacing

\usepackage{titlesec}
\titleformat*{\section}{\headingfont}

\setlrmarginsandblock{1.37in}{*}{*}
\setulmarginsandblock{1.29in}{*}{2}

\checkandfixthelayout
\setmarginnotes{\columnsep}{0.93in}{0.5\onelineskip}

\pretitle{\par\raggedright\Huge\subheadingfont}
\title{\textit{On the Spirit and the Letter}, ch. 4-11, 14-18, 50-54}
\posttitle{ \textperiodcentered} 
\preauthor{\raggedright\Huge\subheadingfont}
\author{ Augustine of Hippo}
\postauthor{\vskip 0.5em}
\predate{}
\date{}
\postdate{}

\newcommand{\gloss}[1]{%
    \marginpar[\RaggedLeft \footnotesize{#1}]{\RaggedRight \footnotesize{#1}}}

\begin{document}
\maketitle
\section*{Chapter 4.— Theirs is a Much More Serious Error, Requiring a Very Vigorous Refutation, Who Deny God's Grace to Be Necessary}


They, however, must be resisted with the utmost ardor and vigor who suppose that without God's help, the mere power of the human will in itself, can either perfect righteousness, or advance steadily towards it; and when they begin to be hard pressed about their presumption in asserting that this result can be reached without the divine assistance, they check themselves, and do not venture to utter such an opinion, because they see how impious and insufferable it is. But they allege that such attainments are not made without God's help on this account, namely, because God both created man with the free choice of his will, and, by giving him commandments, teaches him, Himself, how man ought to live; and indeed assists him, in that He takes away his ignorance by instructing him in the knowledge of what he ought to avoid and to desire in his actions: and thus, by means of the free-will naturally implanted within him, he enters on the way which is pointed out to him, and by persevering in a just and pious course of life, deserves to attain to the blessedness of eternal life.
\section*{Chapter 5 [III.]— True Grace is the Gift of the Holy Ghost, Which Kindles in the Soul the Joy and Love of Goodness}


We, however, on our side affirm that the human will is so divinely aided in the pursuit of righteousness, that (in addition to man's being created with a free-will, and in addition to the teaching by which he is instructed how he ought to live) he receives the Holy Ghost, by whom there is formed in his mind a delight in, and a love of, that supreme and unchangeable good which is God, even now while he is still ``walking by faith'' and not yet ``by sight;'' \gloss{2 Corinthians 5:7} in order that by this gift to him of the earnest, as it were, of the free gift, he may conceive an ardent desire to cleave to his Maker, and may burn to enter upon the participation in that true light, that it may go well with him from Him to whom he owes his existence. A man's free-will, indeed, avails for nothing except to sin, if he knows not the way of truth; and even after his duty and his proper aim shall begin to become known to him, unless he also take delight in and feel a love for it, he neither does his duty, nor sets about it, nor lives rightly. Now, in order that such a course may engage our affections, God's ``love is shed abroad in our hearts,'' not through the free-will which arises from ourselves, but ``through the Holy Ghost, which is given to us.'' \gloss{Romans 5:5} 
\section*{Chapter 6 [IV.]— The Teaching of Law Without the Life-Giving Spirit is The Letter that Kills.}


For that teaching which brings to us the command to live in chastity and righteousness is ``the letter that kills,'' unless accompanied with ``the spirit that gives life.'' For that is not the sole meaning of the passage, ``The letter kills, but the spirit gives life,'' \gloss{2 Corinthians 3:6} which merely prescribes that we should not take in the literal sense any figurative phrase which in the proper meaning of its words would produce only nonsense, but should consider what else it signifies, nourishing the inner man by our spiritual intelligence, since ``being carnally-minded is death, while to be spiritually-minded is life and peace.'' \gloss{Romans 8:6} If, for instance, a man were to take in a literal and carnal sense much that is written in the Song of Solomon, he would minister not to the fruit of a luminous charity, but to the feeling of a libidinous desire. Therefore, the apostle is not to be confined to the limited application just mentioned, when he says, ``The letter kills, but the spirit gives life;'' \gloss{2 Corinthians 3:6} but this is also (and indeed especially) equivalent to what he says elsewhere in the plainest words: ``I had not known lust, except the law had said, You shall not covet;'' \gloss{Romans 7:7} and again, immediately after: ``Sin, taking occasion by the commandment, deceived me, and by it slew me.'' \gloss{Romans 7:11} Now from this you may see what is meant by ``the letter that kills.'' There is, of course, nothing said figuratively which is not to be accepted in its plain sense, when it is said, ``You shall not covet;'' but this is a very plain and salutary precept, and any man who shall fulfil it will have no sin at all. The apostle, indeed, purposely selected this general precept, in which he embraced everything, as if this were the voice of the law, prohibiting us from all sin, when he says, ``You shall not covet;'' for there is no sin committed except by evil concupiscence; so that the law which prohibits this is a good and praiseworthy law. But, when the Holy Ghost withholds His help, which inspires us with a good desire instead of this evil desire (in other words, diffuses love in our hearts), that law, however good in itself, only augments the evil desire by forbidding it. Just as the rush of water which flows incessantly in a particular direction, becomes more violent when it meets with any impediment, and when it has overcome the stoppage, falls in a greater bulk, and with increased impetuosity hurries forward in its downward course. In some strange way the very object which we covet becomes all the more pleasant when it is forbidden. And this is the sin which by the commandment deceives and by it slays, whenever transgression is actually added, which occurs not where there is no law. \gloss{Romans 4:15} 
\section*{Chapter 7 [V.]— What is Proposed to Be Here Treated}


We will, however, consider, if you please, the whole of this passage of the apostle and thoroughly handle it, as the Lord shall enable us. For I want, if possible, to prove that the apostle's words, ``The letter kills, but the spirit gives life,'' do not refer to figurative phrases—although even in this sense a suitable signification might be obtained from them—but rather plainly to the law, which forbids whatever is evil. When I shall have proved this, it will more manifestly appear that to lead a holy life is the gift of God—not only because God has given a free-will to man, without which there is no living ill or well; nor only because He has given him a commandment to teach him how he ought to live; but because through the Holy Ghost He sheds love abroad in the hearts \gloss{Romans 7:7} of those whom he foreknew, in order to predestinate them; whom He predestinated, that He might call them; whom He called, that he might justify them; and whom he justified, that He might glorify them. \gloss{Romans 8:29-30} When this point also shall be cleared, you will, I think, see how vain it is to say that those things only are unexampled possibilities, which are the works of God—such as the passage of the camel through the needle's eye, which we have already referred to, and other similar cases, which to us no doubt are impossible, but easy enough to God; and that man's righteousness is not to be counted in this class of things, on the ground of its being properly man's work, not God's; although there is no reason for supposing, without an example, that his perfection exists, even if it is possible. That these assertions are vain will be clear enough, after it has been also plainly shown that even man's righteousness must be attributed to the operation of God, although not taking place without man's will; and we therefore cannot deny that his perfection is possible even in this life, because all things are possible with God, \gloss{Mark 10:27} — both those which He accomplishes of His own sole will, and those which He appoints to be done with the cooperation with Himself of His creature's will. Accordingly, whatever of such things He does not effect is no doubt without an example in the way of accomplished facts, although with God it possesses both in His power the cause of its possibility, and in His wisdom the reason of its unreality. And should this cause be hidden from man, let him not forget that he is a man; nor charge God with folly simply because he cannot fully comprehend His wisdom.
\section*{Chapter 8.— Romans Interprets Corinthians}


Attend, then, carefully, to the apostle while in his Epistle to the Romans he explains and clearly enough shows that what he wrote to the Corinthians, ``The letter kills, but the spirit gives life,'' \gloss{2 Corinthians 3:6} must be understood in the sense which we have already indicated—that the letter of the law, which teaches us not to commit sin, kills, if the life-giving spirit be absent, forasmuch as it causes sin to be known rather than avoided, and therefore to be increased rather than diminished, because to an evil concupiscense there is now added the transgression of the law.
\section*{Chapter 9 [VI].— Through the Law Sin Has Abounded}


The apostle, then, wishing to commend the grace which has come to all nations through Jesus Christ, lest the Jews should extol themselves at the expense of the other peoples on account of their having received the law, first says that sin and death came on the human race through one man, and that righteousness and eternal life came also through one, expressly mentioning Adam as the former, and Christ as the latter; and then says that ``the law, however, entered, that the offense might abound: but where sin abounded, grace did much more abound: that as sin has reigned unto death, even so might grace reign through righteousness unto eternal life by Jesus Christ our Lord.'' \gloss{Romans 5:20-21} Then, proposing a question for himself to answer, he adds, ``What shall we say then? Shall we continue in sin, that grace may abound? God forbid.''  He saw, indeed, that a perverse use might be made by perverse men of what he had said: ``The law entered, that the offense might abound: but where sin abounded, grace did much more abound,''— as if he had said that sin had been of advantage by reason of the abundance of grace. Rejecting this, he answers his question with a ``God forbid!'' and at once adds: ``How shall we, that are dead to sin, live any longer therein?'' \gloss{Romans 6:2} as much as to say, When grace has brought it to pass that we should die unto sin, what else shall we be doing, if we continue to live in it, than showing ourselves ungrateful to grace? The man who extols the virtue of a medicine does not contend that the diseases and wounds of which the medicine cures him are of advantage to him; on the contrary, in proportion to the praise lavished on the remedy are the blame and horror which are felt of the diseases and wounds healed by the much-extolled medicine. In like manner, the commendation and praise of grace are vituperation and condemnation of offenses. For there was need to prove to man how corruptly weak he was, so that against his iniquity, the holy law brought him no help towards good, but rather increased than diminished his iniquity; seeing that the law entered, that the offense might abound; that being thus convicted and confounded, he might see not only that he needed a physician, but also God as his helper so to direct his steps that sin should not rule over him, and he might be healed by betaking himself to the help of the divine mercy; and in this way, where sin abounded grace might much more abound—not through the merit of the sinner, but by the intervention of his Helper.
\section*{Chapter 10.— Christ the True Healer}


Accordingly, the apostle shows that the same medicine was mystically set forth in the passion and resurrection of Christ, when he says, ``Do you not know, that so many of us as were baptized into Jesus Christ were baptized into His death? Therefore we were buried with Him by baptism into death; that like as Christ was raised up from the dead by the glory of the Father, even so we also should walk in newness of life. For if we have been planted together in the likeness of His death, we shall be also in the likeness of His resurrection: knowing this, that our old man is crucified with Him, that the body of sin might be destroyed, that henceforth we should not serve sin. For he that is dead is justified from sin. Now, if we be dead with Christ, we believe that we shall also live with Him: knowing that Christ, being raised from the dead, dies no more; death has no more dominion over Him. For in that He died, He died unto sin once; but in that He lives, He lives unto God. Likewise reckon ye also yourselves to be dead indeed unto sin, but alive unto God through Jesus Christ our Lord.'' \gloss{Romans 6:3-11} Now it is plain enough that here by the mystery of the Lord's death and resurrection is figured the death of our old sinful life, and the rising of the new; and that here is shown forth the abolition of iniquity and the renewal of righteousness. Whence then arises this vast benefit to man through the letter of the law, except it be through the faith of Jesus Christ?
\section*{Chapter 11 [VII.]— From What Fountain Good Works Flow}


This holy meditation preserves ``the children of men, who put their trust under the shadow of God's wings,''  so that they are ``drunken with the fatness of His house, and drink of the full stream of His pleasure. For with Him is the fountain of life, and in His light shall they see light. For He extends His mercy to them that know Him, and His righteousness to the upright in heart.''  He does not, indeed, extend His mercy to them because they know Him, but that they may know Him; nor is it because they are upright in heart, but that they may become so, that He extends to them His righteousness, whereby He justifies the ungodly. \gloss{Romans 4:5} This meditation does not elevate with pride: this sin arises when any man has too much confidence in himself, and makes himself the chief end of living. Impelled by this vain feeling, he departs from that fountain of life, from the draughts of which alone is imbibed the holiness which is itself the good life—and from that unchanging light, by sharing in which the reasonable soul is in a certain sense inflamed, and becomes itself a created and reflected luminary; even as ``John was a burning and a shining light,'' \gloss{John 5:35} who notwithstanding acknowledged the source of his own illumination in the words, ``Of His fullness have all we received.'' \gloss{John 1:16} Whose, I would ask, but His, of course, in comparison with whom John indeed was no light at all? For ``that was the true light, which lights every man that comes into the world.'' \gloss{John 1:9} Therefore, in the same psalm, after saying, ``Extend Your mercy to them that know You, and Your righteousness to the upright in heart,''  he adds, ``Let not the foot of pride come against me, and let not the hands of sinners move me. There have fallen all the workers of iniquity: they are cast out, and are not able to stand.''  Since by that impiety which leads each to attribute to himself the excellence which is God's, he is cast out into his own native darkness, in which consist the works of iniquity. For it is manifestly these works which he does, and for the achievement of such alone is he naturally fit. The works of righteousness he never does, except as he receives ability from that fountain and that light, where the life is that wants for nothing, and where is ``no variableness, nor the shadow of turning.'' \gloss{James 1:17} 
\section*{Chapter 14.— In What Respect the Pelagians Acknowledge God as the Author of Our Justification}


``But,'' say they, ``we do praise God as the Author of our righteousness, in that He gave the law, by the teaching of which we have learned how we ought to live.'' But they give no heed to what they read: ``By the law there shall no flesh be justified in the sight of God.'' \gloss{Romans 3:20} This may indeed be possible before men, but not before Him who looks into our very heart and inmost will, where He sees that, although the man who fears the law keeps a certain precept, he would nevertheless rather do another thing if he were permitted. And lest any one should suppose that, in the passage just quoted from him, the apostle had meant to say that none are justified by that law, which contains many precepts, under the figure of the ancient sacraments, and among them that circumcision of the flesh itself, which infants were commanded to receive on the eighth day after birth; he immediately adds what law he meant, and says, ``For by the law is the knowledge of sin.'' \gloss{Romans 3:20} He refers then to that law of which he afterwards declares, ``I had not known sin but by the law; for I had not known lust except the law had said, You shall not covet.'' \gloss{Romans 7:7} For what means this but that ``by the law comes the knowledge of sin?''
\section*{Chapter 15 [IX.]— The Righteousness of God Manifested by the Law and the Prophets}


Here, perhaps, it may be said by that presumption of man, which is ignorant of the righteousness of God, and wishes to establish one of its own, that the apostle quite properly said, ``For by the law shall no man be justified,'' \gloss{Romans 3:20} inasmuch as the law merely shows what one ought to do, and what one ought to guard against, in order that what the law thus points out may be accomplished by the will, and so man be justified, not indeed by the power of the law, but by his free determination. But I ask your attention, O man, to what follows. ``But now the righteousness of God,'' says he, ``without the law is manifested, being witnessed by the law and the prophets.'' \gloss{Romans 3:21} Does this then sound a light thing in deaf ears? He says, ``The righteousness of God is manifested.'' Now this righteousness they are ignorant of, who wish to establish one of their own; they will not submit themselves to it. \gloss{Romans 10:3} His words are, `` \textit{The righteousness of God} is manifested:'' he does not say, the righteousness of man, or the righteousness of his own will, but the ``righteousness \textit{of God},''— not that whereby He is Himself righteous, but that with which He endows man when He justifies the ungodly. This is witnessed by the law and the prophets; in other words, the law and the prophets each afford it testimony. The law, indeed, by issuing its commands and threats, and by justifying no man, sufficiently shows that it is by God's gift, through the help of the Spirit, that a man is justified; and the prophets, because it was what they predicted that Christ at His coming accomplished. Accordingly he advances a step further, and adds, ``But righteousness of God by faith of Jesus Christ,'' \gloss{Romans 3:22} that is by the faith wherewith one believes in Christ for just as there is not meant the faith with which Christ Himself believes, so also there is not meant the righteousness whereby God is Himself righteous. Both no doubt are ours, but yet they are called God's, and Christ's, because it is by their bounty that these gifts are bestowed upon us. The righteousness of God then is without the law, but not manifested without the law; for if it were manifested without the law, how could it be witnessed by the law? That righteousness of God, however, is without the law, which God by the Spirit of grace bestows on the believer without the help of the law,— that is, when not helped by the law. When, indeed, He by the law discovers to a man his weakness, it is in order that by faith he may flee for refuge to His mercy, and be healed. And thus concerning His wisdom we are told, that ``she carries law and mercy upon her tongue,'' \gloss{Proverbs 3:16} — the `` \textit{law},'' whereby she may convict the proud, the `` \textit{mercy},'' wherewith she may justify the humbled. ``The righteousness of God,'' then, ``by faith of Jesus Christ, is unto all that believe; for there is no difference, for all have sinned, and come short of the glory of God'' \gloss{Romans 3:22-23} — not of their own glory. For what have they, which they have not received? Now if they received it, why do they glory as if they had not received it? \gloss{1 Corinthians 4:7} Well, then, they come short of the glory of God; now observe what follows: ``Being justified freely by His grace.'' \gloss{Romans 3:24} It is not, therefore, by the law, nor is it by their own will, that they are justified; but they are justified \textit{freely by His grace}—not that it is wrought without our will; but our will is by the law shown to be weak, that grace may heal its infirmity; and that our healed will may fulfil the law, not by compact under the law, nor yet in the absence of law.
\section*{Chapter 16 [X.]— How the Law Was Not Made for a Righteous Man}


Because ``for a righteous man the law was not made;'' \gloss{1 Timothy 1:8} and yet ``the law is good, if a man use it lawfully.'' \gloss{1 Timothy 1:9} Now by connecting together these two seemingly contrary statements, the apostle warns and urges his reader to sift the question and solve it too. For how can it be that ``the law is good, if a man use it lawfully,'' if what follows is also true: ``Knowing this, that the law is not made for a righteous man?'' \gloss{1 Timothy 1:9} For who but a righteous man lawfully uses the law? Yet it is not for him that it is made, but for the unrighteous. Must then the unrighteous man, in order that he may be justified,— that is, become a righteous man—lawfully use the law, to lead him, as by the schoolmaster's hand, \gloss{Galatians 3:24} to that grace by which alone he can fulfil what the law commands? Now it is freely that he is justified thereby—that is, on account of no antecedent merits of his own works; ``otherwise grace is no more grace,'' \gloss{Romans 11:6} since it is bestowed on us, not because we have done good works, but that we may be able to do them—in other words, not because we have fulfilled the law, but in order that we may be able to fulfil the law. Now He said, ``I am not come to destroy the law, but to fulfil it,'' \gloss{Matthew 5:17} of whom it was said, ``We have seen His glory, the glory as of the only-begotten of the Father, full of grace and truth.'' \gloss{John 1:14} This is the glory which is meant in the words, ``All have sinned, and come short of the glory of God;'' \gloss{Romans 3:23} and this the grace of which he speaks in the next verse, ``Being justified freely by His grace.'' \gloss{Romans 3:24} The unrighteous man therefore lawfully uses the law, that he may become righteous; but when he has become so, he must no longer use it as a chariot, for he has arrived at his journey's end—or rather (that I may employ the apostle's own simile, which has been already mentioned) as a schoolmaster, seeing that he is now fully learned. How then is the law not made for a righteous man, if it is necessary for the righteous man too, not that he may be brought as an unrighteous man to the grace that justifies, but that he may use it lawfully, now that he is righteous? Does not the case perhaps stand thus—nay, not \textit{perhaps}, but rather \textit{ certainly},— that the man who has become righteous thus lawfully uses the law, when he applies it to alarm the unrighteous, so that whenever the disease of some unusual desire begins in them, too, to be augmented by the incentive of the law's prohibition and an increased amount of transgression, they may in faith flee for refuge to the grace that justifies, and becoming delighted with the sweet pleasures of holiness, may escape the penalty of the law's menacing letter through the spirit's soothing gift? In this way the two statements will not be contrary, nor will they be repugnant to each other: even the righteous man may lawfully use a good law, and yet the law be not made for the righteous man; for it is not by the law that he becomes righteous, but by the law of faith, which led him to believe that no other resource was possible to his weakness for fulfilling the precepts which ``the law of works'' \gloss{Romans 3:27} commanded, except to be assisted by the grace of God.
\section*{Chapter 17.— The Exclusion of Boasting}


Accordingly he says, ``Where is boasting then? It is excluded. By what law? Of works? Nay; but by the law of faith.'' \gloss{Romans 3:27} He may either mean, the laudable boasting, which is in the Lord; and that it is \textit{excluded}, not in the sense that it is driven off so as to pass away, but that it is clearly manifested so as to stand out prominently. Whence certain artificers in silver are called `` \textit{exclusores}.''  In this sense it occurs also in that passage in the Psalms: ``That they may be \textit{excluded}, who have been proved with silver,'' — that is, that they may stand out in prominence, who have been tried by the word of God. For in another passage it is said: ``The words of the Lord are pure words, as silver which is tried in the fire.''  Or if this be not his meaning, he must have wished to mention that vicious boasting which comes of pride— that is, of those who appear to themselves to lead righteous lives, and boast of their excellence as if they had not received it—and further to inform us, that by the law of faith, not by the law of works, this boasting was \textit{excluded}, in the other sense of shut out and driven away; because by the law of faith every one learns that whatever good life he leads he has from the grace of God, and that from no other source whatever can he obtain the means of becoming perfect in the love of righteousness.
\section*{Chapter 18 [XI.]— Piety is Wisdom; That is Called the Righteousness of God, Which He Produces}


Now, this meditation makes a man godly, and this godliness is true wisdom. By godliness I mean that which the Greeks designate \greektext{θεοσέβεια}, — that very virtue which is commended to man in the passage of Job, where it is said to him, ``Behold, godliness is wisdom.'' \gloss{Job 28:28} Now if the word \greektext{θεοσέβεια} be interpreted according to its derivation, it might be called `` \textit{the worship of God};''  and in this worship the essential point is, that the soul be not ungrateful to Him. Whence it is that in the most true and excellent sacrifice we are admonished to ``give thanks unto our Lord God.''  Ungrateful however, our soul would be, were it to attribute to itself that which it received from God, especially the righteousness, with the works of which (the special property, as it were, of itself, and produced, so to speak, by the soul itself for itself) it is not puffed up in a vulgar pride, as it might be with riches, or beauty of limb, or eloquence, or those other accomplishments, external or internal, bodily or mental, which wicked men too are in the habit of possessing, but, if I may say so, in a wise complacency, as of things which constitute in a special manner the good works of the good. It is owing to this sin of vulgar pride that even some great men have drifted from the sure anchorage of the divine nature, and have floated down into the shame of idolatry. Whence the apostle again in the same epistle, wherein he so firmly maintains the principle of grace, after saying that he was a debtor both to the Greeks and to the Barbarians, to the wise and to the unwise, and professing himself ready, so far as to him pertained, to preach the gospel even to those who lived in Rome, adds: ``I am not ashamed of the Gospel of Christ: for it is the power of God unto salvation to every one that believes; to the Jew first, and also to the Greek. For therein is the righteousness of God revealed from faith to faith: as it is written, The just shall live by faith.'' \gloss{Romans 1:14-17} This is the righteousness of God, which was veiled in the Old Testament, and is revealed in the New; and it is called \textit{the righteousness of God}, because by His bestowal of it He makes us righteous, just as we read that ``salvation is the Lord's,''  because He makes us safe. And this is the faith ``from which'' and ``to which'' it is revealed,— \textit{from the faith} of them who preach it, \textit{to the faith} of those who obey it. By this faith of Jesus Christ— that is, the faith which Christ has given to us— we believe it is from God that we now have, and shall have more and more, the ability of living righteously; wherefore we give Him thanks with that dutiful worship with which He only is to be worshipped.
\section*{Chapter 50 [XXIX.]— Righteousness is the Gift of God}


Let no man therefore boast of that which he seems to possess, as if he had not received it; \gloss{1 Corinthians 4:7} nor let him think that he has received it merely because the external letter of the law has been either exhibited to him to read, or sounded in his ear for him to hear. For ``if righteousness is by the law, then Christ has died in vain.'' \gloss{Galatians 2:21} Seeing, however, that if He has not died in vain, He has ascended up on high, and has led captivity captive, and has given gifts to men,  it follows that whosoever has, has from this source. But whosoever denies that he has from Him, either has not, or is in great danger of being deprived of what he has.  ``For it is one God which justifies the circumcision by faith, and the uncircumcision through faith;'' \gloss{Romans 3:30} in which clauses there is no real difference in the sense, as if the phrase `` \textit{by faith}'' meant one thing, and `` \textit{through faith}'' another, but only a variety of expression. For in one passage, when speaking of the Gentiles—that is, of the uncircumcision,— he says, ``The Scripture, foreseeing that God would justify the heathen \textit{by faith};'' \gloss{Galatians 3:8} and again, in another, when speaking of the circumcision, to which he himself belonged, he says, ``We who are Jews by nature, and not sinners of the Gentiles, knowing that a man is not justified by the works of the law, but \textit{through faith} in Jesus Christ, even we believed in Jesus Christ.''  Observe, he says that both the uncircumcision are justified by faith, and the circumcision through faith, if, indeed, the circumcision keep the righteousness of faith. For the Gentiles, which followed not after righteousness, have attained to righteousness, even the righteousness which is by faith, \gloss{Romans 9:30} — by obtaining it of God, not by assuming it of themselves. But Israel, which followed after the law of righteousness, has not attained to the law of righteousness. And why? Because they sought it not by faith, but as it were by works \gloss{Romans 9:31-32} — in other words, working it out as it were by themselves, not believing that it is God who works within them. ``For it is God which works in us both to will and to do of His own good pleasure.'' \gloss{Philippians 2:13} And hereby ``they stumbled at the stumbling-stone.'' \gloss{Romans 9:32} For what he said, ``not by faith, but as it were by works,'' \gloss{Romans 9:32} he most clearly explained in the following words: ``They, being ignorant of God's righteousness, and going about to establish their own righteousness, have not submitted themselves unto the righteousness of God. For Christ is the end of the law for righteousness to every one that believes.'' \gloss{Romans 10:3-4} Then are we still in doubt what are those works of the law by which a man is not justified, if he believes them to be his own works, as it were, without the help and gift of God, which is ``by the faith of Jesus Christ?'' And do we suppose that they are circumcision and the other like ordinances, because some such things in other passages are read concerning these sacramental rites too? In this place, however, it is certainly not circumcision which they wanted to establish as their own righteousness, because God established this by prescribing it Himself. Nor is it possible for us to understand this statement, of those works concerning which the Lord says to them, ``You reject the commandment of God, that you may keep your own tradition;'' \gloss{Mark 7:9} because, as the apostle says, Israel, which followed after the law of righteousness, has not attained to the law of righteousness. \gloss{Romans 9:31} He did not say, Which followed after their own traditions, framing them and relying on them. This then is the sole distinction, that the very precept, ``You shall not covet,'' \gloss{Exodus 20:17} and God's other good and holy commandments, they attributed to themselves; whereas, that man may keep them, God must work in him through faith in Jesus Christ, who is ``the end of the law for righteousness to every one that believes.'' \gloss{Romans 10:4} That is to say, every one who is incorporated into Him and made a member of His body, is able, by His giving the increase within, to work righteousness. It is of such a man's works that Christ Himself has said, ``Without me you can do nothing.'' \gloss{John 15:5} 
\section*{Chapter 51.— Faith the Ground of All Righteousness}


The righteousness of the law is proposed in these terms—that whosoever shall do it shall live in it; and the purpose is, that when each has discovered his own weakness, he may not by his own strength, nor by the letter of the law (which cannot be done), but by faith, conciliating the Justifier, attain, and do, and live in it. For the work in which he who does it shall live, is not done except by one who is justified. His justification, however, is obtained by faith; and concerning faith it is written, ``Say not in your heart, Who shall ascend into heaven? (that is, to bring down Christ therefrom;) or, Who shall descend into the deep? (that is, to bring up Christ again from the dead.) But what says it? The word is near you, even in your mouth, and in your heart: that is (says he), the word of faith which we preach: That if you shall confess with your mouth the Lord Jesus, and shall believe in your heart that God has raised Him from the dead, you shall be saved.'' \gloss{Romans 10:6-9} As far as he is saved, so far is he righteous. For by this faith we believe that God will raise even us from the dead—even now in the spirit, that we may in this present world live soberly, righteously, and godly in the renewal of His grace; and by and by in our flesh, which shall rise again to immortality, which indeed is the reward of the Spirit, who precedes it by a resurrection which is appropriate to Himself—that is, by justification. ``For we are buried with Christ by baptism unto death, that like as Christ was raised up from the dead by the glory of the Father, even so we also should walk in newness of life.'' \gloss{Romans 6:4} By faith, therefore, in Jesus Christ we obtain salvation—both in so far as it is begun within us in reality, and in so far as its perfection is waited for in hope; ``for whosoever shall call on the name of the Lord shall be saved.''  ``How abundant,'' says the Psalmist, ``is the multitude of Your goodness, O Lord, which You have laid up for them that fear You, and hast perfected for them that hope in You!''  By the law we fear God; by faith we hope in God: but from those who fear punishment grace is hidden. And the soul which labours under this fear, since it has not conquered its evil concupiscence, and from which this fear, like a harsh master, has not departed—let it flee by faith for refuge to the mercy of God, that He may give it what He commands, and may, by inspiring into it the sweetness of His grace through His Holy Spirit, cause the soul to delight more in what He teaches it, than it delights in what opposes His instruction. In this manner it is that the great abundance of His sweetness—that  is, the law of faith—His love which is in our hearts, and shed abroad, is perfected in them that hope in Him, that good may be wrought by the soul, healed not by the fear of punishment, but by the love of righteousness.
\section*{Chapter 52 [XXX.]— Grace Establishes Free Will}


Do we then by grace make void free will? God forbid! Nay, rather we establish free will. For even as the law by faith, so free will by grace, is not made void, but established. \gloss{Romans 3:31} For neither is the law fulfilled except by free will; but by the law is the knowledge of sin, by faith the acquisition of grace against sin, by grace the healing of the soul from the disease of sin, by the health of the soul freedom of will, by free will the love of righteousness, by love of righteousness the accomplishment of the law. Accordingly, as the law is not made void, but is established through faith, since faith procures grace whereby the law is fulfilled; so free will is not made void through grace, but is established, since grace cures the will whereby righteousness is freely loved. Now all the stages which I have here connected together in their successive links, have severally their proper voices in the sacred Scriptures. The law says: ``You shall not covet.'' \gloss{Exodus 20:17} Faith says: ``Heal my soul, for I have sinned against You.''  Grace says: ``Behold, you are made whole: sin no more, lest a worse thing come unto you.'' \gloss{John 5:14} Health says: ``O Lord my God, I cried unto You, and You have healed me.''  Free will says: ``I will freely sacrifice unto You.''  Love of righteousness says: ``Transgressors told me pleasant tales, but not according to Your law, O Lord.''  How is it then that miserable men dare to be proud, either of their free will, before they are freed, or of their own strength, if they have been freed? They do not observe that in the very mention of free will they pronounce the name of liberty. But ``where the Spirit of the Lord is, there is liberty.'' \gloss{2 Corinthians 3:17} If, therefore, they are the slaves of sin, why do they boast of free will? For by what a man is overcome, to the same is he delivered as a slave. \gloss{2 Peter 2:19} But if they have been freed, why do they vaunt themselves as if it were by their own doing, and boast, as if they had not received? Or are they free in such sort that they do not choose to have Him for their Lord who says to them: ``Without me you can do nothing;'' \gloss{John 15:5} and ``If the Son shall make you free, you shall be free indeed?'' \gloss{John 8:36} 
\section*{Chapter 53 [XXXI.]— Volition and Ability}


Some one will ask whether the faith itself, in which seems to be the beginning either of salvation, or of that series leading to salvation which I have just mentioned, is placed in our power. We shall see more easily, if we first examine with some care what ``our power'' means. Since, then, there are two things—will and ability; it follows that not every one that has the will has therefore the ability also, nor has every one that possesses the ability the will also; for as we sometimes will what we cannot do, so also we sometimes can do what we do not will. From the words themselves when sufficiently considered, we shall detect, in the very ring of the terms, the derivation of volition from willingness, and of \textit{ability} from ableness.  Therefore, even as the man who wishes has volition, so also the man who can has ability. But in order that a thing may be done by ability, the volition must be present. For no man is usually said to do a thing with ability if he did it unwillingly. Although, at the same time, if we observe more precisely, even what a man is compelled to do unwillingly, he does, if he does it, by his volition; only he is said to be an unwilling agent, or to act against his will, because he would prefer some other thing. He is compelled, indeed, by some unfortunate influence, to do what he does under compulsion, wishing to escape it or to remove it out of his way. For if his volition be so strong that he prefers not doing this to not suffering that, then beyond doubt he resists the compelling influence, and does it not. And accordingly, if he does it, it is not with a full and free will, but yet it is not without will that he does it; and inasmuch as the volition is followed by its effect, we cannot say that he lacked the ability to do it. If, indeed, he willed to do it, yielding to compulsion, but could not, although we should allow that a coerced will was present, we should yet say that ability was absent. But when he did not do the thing because he was unwilling, then of course the ability was present, but the volition was absent, since he did it not, by his resistance to the compelling influence. Hence it is that even they who compel, or who persuade, are accustomed to say, Why don't you do what you have in your ability, in order to avoid this evil? While they who are utterly unable to do what they are compelled to do, because they are supposed to be able usually answer by excusing themselves, and say, I would do it if it were in my ability. What then do we ask more, since we call that ability when to the volition is added the faculty of doing? Accordingly, every one is said to have that in his ability which he does if he likes, and does not if he dislikes.
\section*{Chapter 54.— Whether Faith Be in a Man's Own Power}


Attend now to the point which we have laid down for discussion: whether faith is in our own power? We now speak of that faith which we employ when we believe anything, not that which we give when we make a promise; for this too is called .  We use the word in one sense when we say, ``He had no faith in me,'' and in another sense when we say, ``He did not keep faith with me.'' The one phrase means, ``He did not believe what I said;'' the other, ``He did not do what he promised.'' According to the faith by which we believe, we are faithful to God; but according to that whereby a thing is brought to pass which is promised, God Himself even is faithful to us; for the apostle declares, ``God is faithful, who will not suffer you to be tempted above that you are able.'' \gloss{1 Corinthians 10:13} Well, now, the former is the faith about which we inquire, Whether it be in our power? Even the faith by which we believe God, or believe in God. For of this it is written, ``Abraham believed God, and it was counted unto him for righteousness.''  And again, ``To him that believes in Him that justifies the ungodly, his faith is counted for righteousness.'' \gloss{Romans 4:5} Consider now whether anybody believes, if he be unwilling; or whether he believes not, if he shall have willed it. Such a position, indeed, is absurd (for what is believing but consenting to the truth of what is said? And this consent is certainly voluntary): faith, therefore, is in our own power. But, as the apostle says: ``There is no power but comes from God,'' \gloss{Romans 13:1} what reason then is there why it may not be said to us even of this: ``What have you which you have not received?'' \gloss{1 Corinthians 4:7} — for it is God who gave us even to believe. Nowhere, however, in Holy Scripture do we find such an assertion as, There is no volition but comes from God. And rightly is it not so written, because it is not true: otherwise God would be the author even of sins (which Heaven forbid!), if there were no volition except what comes from Him; inasmuch as an evil volition alone is already a sin, even if the effect be wanting—in other words, if it has not ability. But when the evil volition receives ability to accomplish its intention, this proceeds from the judgment of God, with whom there is no unrighteousness. \gloss{Romans 9:14} He indeed punishes after this manner; nor is His chastisement unjust because it is secret. The ungodly man, however, is not aware that he is being punished, except when he unwillingly discovers by an open penalty how much evil he has willingly committed. This is just what the apostle says of certain men: ``God has given them up to the evil desires of their own hearts, . . .to do those things that are not convenient.''  Accordingly, the Lord also said to Pilate: ``You could have no power at all against me, except it were given you from above.'' \gloss{John 19:11} But still, when the ability is given, surely no necessity is imposed. Therefore, although David had received ability to kill Saul, he preferred sparing to striking him.  Whence we understand that bad men receive ability for the condemnation of their depraved will, while good men receive ability for trying of their good will.
\end{document}
