\documentclass[9pt, twocolumn, oneside, a4paper]{memoir}
\usepackage{fontspec}
\setmainfont{Alegreya}
\usepackage{ragged2e}
\newfontfamily\scshape{Alegreya SC}
\newfontfamily\headingfont{Alegreya Sans SC}
\newfontfamily\subheadingfont[
    BoldFont={Alegreya Sans Medium},
    ItalicFont={Alegreya Sans Light Italic},
    BoldItalicFont={Alegreya Sans Medium Italic}
]{Alegreya Sans Light}
\newfontfamily\greekfont[Script=Greek, Scale=1.14, WordSpace=0.7]{GFS Neohellenic}
\usepackage{polyglossia}
\setdefaultlanguage{english}
\setotherlanguage[variant=ancient]{greek}
\newcommand{\greektext}[1]{\foreignlanguage{greek}{#1}}

\usepackage{microtype}
\emergencystretch \hsize
\tolerance 300%
\setlength{\parskip}{0pt}
\frenchspacing

\usepackage{titlesec}
\titleformat*{\section}{\headingfont}

\setlrmarginsandblock{1.37in}{*}{*}
\setulmarginsandblock{1.29in}{*}{2}

\checkandfixthelayout
\setmarginnotes{\columnsep}{0.93in}{0.5\onelineskip}

\pretitle{\par\raggedright\Huge\subheadingfont}
\title{\textit{On the Incarnation of the Word}, chs. 1-3}
\posttitle{ \textperiodcentered} 
\preauthor{\raggedright\Huge\subheadingfont}
\author{ Athanasius of Alexandria}
\postauthor{\vskip 0.5em}
\predate{}
\date{}
\postdate{}

\newcommand{\gloss}[1]{%
    \marginpar[\RaggedLeft \footnotesize{#1}]{\RaggedRight \footnotesize{#1}}}

\begin{document}
\maketitle
\section*{Chapter 1}



\textsc{(1) }In our former book       we dealt fully enough with a few of the chief points about the       heathen worship of idols, and how those false fears originally arose. We also,       by God's grace, briefly indicated that the Word of the Father is Himself divine,       that all things that are owe their being to His will and power, and that it is       through Him that the Father gives order to creation, by Him that all things are       moved, and through Him that they receive their being. Now, Macarius, true lover       of Christ, we must take a step further in the faith of our holy religion, and       consider also the Word's becoming Man and His divine Appearing in our midst.       That mystery the Jews traduce, the Greeks deride, but we adore; and your own       love and devotion to the Word also will be the greater, because in His Manhood       He seems so little worth. For it is a fact that the more unbelievers pour scorn       on Him, so much the more does He make His Godhead evident. The things which       they, as men, rule out as impossible, He plainly shows to be possible; that       which they deride as unfitting, His goodness makes most fit; and things which       these wiseacres laugh at as "human" He by His inherent might declares divine.       Thus by what seems His utter poverty and weakness on the cross He overturns the       pomp and parade of idols, and quietly and hiddenly wins over the mockers and       unbelievers to recognize Him as God.   

Now in dealing with these matters it is necessary first to recall what has       already been said. You must understand why it is that the Word of the Father, so       great and so high, has been made manifest in bodily form. He has not assumed a       body as proper to His own nature, far from it, for as the Word He is without       body. He has been manifested in a human body for this reason only, out of the       love and goodness of His Father, for the salvation of us men. We will begin,       then, with the creation of the world and with God its Maker, for the first fact       that you must grasp is this: . There is thus no inconsistency between creation and salvation for the One Father has       employed the same Agent for both works, effecting the salvation of the world       through the       same Word Who made it in the beginning.   

\textsc{(2)} In regard to the making of the universe and the       creation of all things there have been various opinions, and each person has       propounded the theory that suited his own taste. For instance, some say that all       things are self-originated and, so to speak, haphazard. The Epicureans are among       these; they deny that there is any Mind behind the universe at all. This view is       contrary to all the facts of experience, their own existence included. For if       all things had come into being in this automatic fashion, instead of being the       outcome of Mind, though they existed, they would all be uniform and without       distinction. In the universe everything would be sun or moon or whatever it was,       and in the human body the whole would be hand or eye or foot. But in point of       fact the sun and the moon and the earth are all different things, and even       within the human body there are different members, such as foot and hand and       head. This distinctness of things argues not a spontaneous generation but a       prevenient Cause; and from that Cause we can apprehend God, the Designer and       Maker of all.   

Others take the view expressed by Plato, that giant among the Greeks. He said       that God had made all things out of pre-existent and uncreated matter, just as       the carpenter makes things only out of wood that already exists. But those who       hold this view do not realize that to deny that God is Himself the Cause of       matter is to impute limitation to Him, just as it is undoubtedly a limitation on       the part of the carpenter that he can make nothing unless he has the wood. How       could God be called Maker and Artificer if His ability to make depended on some       other cause, namely on matter itself? If He only worked up existing matter and       did not Himself bring matter into being, He would be not the Creator but only a       craftsman.   

Then, again, there is the theory of the Gnostics, who have invented for       themselves an Artificer of all things other than the Father of our Lord Jesus       Christ. These simply shut their eyes to the obvious meaning of Scripture. For       instance, the Lord, having reminded the Jews of the statement in Genesis, "He Who created them in the beginning made them      male       and female . . . ," and having shown that for that reason a man should leave       his parents and cleave to his wife, goes on to say with reference to the       Creator, "What therefore God has joined together, let no man put asunder."       How       can they get a creation independent of the Father out of that? And, again, St.       John, speaking all inclusively, says,       "All things became by Him and without Him came nothing       into being."       How then could the Artificer be someone different, other than the Father of Christ?   

\textsc{(3)} Such are the notions which men put forward. But the       impiety of their foolish talk is plainly declared by the divine teaching of the       Christian faith. From it we know that, because there is Mind behind the       universe, it did not originate itself; because God is infinite, not finite, it       was not made from pre-existent matter, but out of nothing and out of       non-existence absolute and utter God brought it into being through the Word. He       says as much in Genesis:             "In the beginning God created the heavens and the       earth;       and again through that most helpful book ,       "Believe thou first and foremost that there is One God       Who created and arranged all things and brought them out of non-existence into       being."       Paul also indicates the same thing when he says,       "By faith we understand that the worlds were framed by       the Word of God, so that the things which we see now did not come into being       out of things which had previously appeared."       For God is good—or rather, of all goodness He is Fountainhead, and it is impossible       for one who is good to be mean or grudging about anything. Grudging existence to       none therefore, He made all things out of nothing through His own Word, our Lord       Jesus Christ and of all these His earthly creatures He reserved especial mercy       for the race of men. Upon them, therefore, upon men who, as animals, were       essentially impermanent, He bestowed a grace which other creatures       lacked—namely the impress of His own Image, a share in the reasonable being of       the very Word Himself, so that, reflecting Him and themselves becoming       reasonable and expressing the Mind of God even as He does, though in limited       degree they might continue for ever in the blessed and only true life of the       saints in paradise. But since the will of man could turn either way, God secured       this grace that He had given by making it conditional from the first upon two       things—namely, a law and a place. He set them in His own paradise, and laid       upon them a single prohibition. If they guarded the grace and retained the       loveliness of their original innocence, then the life of paradise should be       theirs, without sorrow, pain or care, and after it the assurance of immortality       in heaven. But if they went astray and became vile, throwing away their       birthright of beauty, then they would come under the natural law of death and       live no longer in paradise, but, dying outside of it, continue in death and in       corruption. This is what Holy Scripture tells us, proclaiming the command of God,       "Of every tree that is in the garden thou shalt surely       eat, but of the tree of the knowledge of good and evil ye shall not eat, but       in the day that ye do eat, ye shall surely die."       —not just die only, but remain in the state of death       and of corruption.   

\textsc{(4)} You may be wondering why we are discussing the       origin of men when we set out to talk about the Word's becoming Man. The former       subject is relevant to the latter for this reason: it was our sorry case that       caused the Word to come down, our transgression that called out His love for us,       so that He made haste to help us and to appear among us. It is we who were the       cause of His taking human form, and for our salvation that in His great love He       was both born and manifested in a human body. For God had made man thus (that       is, as an embodied spirit), and had willed that he should remain in       incorruption. But men, having turned from the contemplation of God to evil of       their own devising, had come inevitably under the law of death. Instead of       remaining in the state in which God had created them, they were in process of       becoming corrupted entirely, and death had them completely under its dominion.       For the       transgression of the commandment was making them turn back again       according to their nature; and as they had at the beginning come into being out       of non-existence, so were they now on the way to returning, through corruption,       to non-existence again. The presence and love of the Word had called them into       being; inevitably, therefore when they lost the knowledge of God, they lost       existence with it; for it is God alone Who exists, evil is non-being, the       negation and antithesis of good. By nature, of course, man is mortal, since he       was made from nothing; but he bears also the Likeness of Him Who is, and if he       preserves that Likeness through constant contemplation, then his nature is       deprived of its power and he remains incorrupt. So is it affirmed in Wisdom:       "The keeping of His laws is the assurance of       incorruption."       And being incorrupt, he would be henceforth as God, as Holy Scripture says,       "I have said, Ye are gods and sons of the Highest all       of you: but ye die as men and fall as one of the princes."

\textsc{(5)} This, then, was the plight of men. God had not only       made them out of nothing, but had also graciously bestowed on them His own life       by the grace of the Word. Then, turning from eternal things to things       corruptible, by counsel of the devil, they had become the cause of their own       corruption in death; for, as I said before, though they were by nature subject       to corruption, the grace of their union with the Word made them capable of       escaping from the natural law, provided that they retained the beauty of       innocence with which they were created. That is to say, the presence of the Word       with them shielded them even from natural corruption, as also Wisdom says:       "God created man for incorruption and as an image of       His own eternity; but by envy of the devil death entered into the world."       When this happened, men began to die, and corruption ran riot among them and held       sway over them to an even more than natural degree, because it was the penalty       of which God had forewarned them for transgressing the commandment. Indeed, they       had in their sinning surpassed all limits; for, having invented wickedness in       the beginning and so involved themselves in death and corruption, they had gone       on gradually from bad to worse, not stopping at any one kind of evil, but       continually, as with insatiable appetite, devising new kinds of sins. Adulteries       and thefts were everywhere, murder and raping filled the earth, law was       disregarded in corruption and injustice, all kinds of iniquities were       perpetrated by all, both singly and in common. Cities were warring with cities,       nations were rising against nations, and the whole earth was rent with factions       and battles, while each strove to outdo the other in wickedness. Even crimes       contrary to nature were not unknown, but as the martyr-apostle of Christ says:       "Their women changed the natural use into that which       is against nature; and the men also, leaving the natural use of the woman,       flamed out in lust towards each other, perpetrating shameless acts with their       own sex, and receiving in their own persons the due recompense of their       pervertedness."

\section*{Chapter 2}



\textsc{(6)} We saw in the last chapter that, because death and       corruption were gaining ever firmer hold on them, the human race was in process       of destruction. Man, who was created in God's image and in his possession of       reason reflected the very Word Himself, was disappearing, and the work of God       was being undone. The law of death, which followed from the Transgression,       prevailed upon us, and from it there was no escape. The thing that was happening       was in truth both monstrous and unfitting. It would, of course, have been       unthinkable that God should go back upon His word and that man, having       transgressed, should not die; but it was equally monstrous that beings which       once had shared the nature of the Word should perish and turn back again into       non-existence through corruption. It was unworthy of the goodness of God that       creatures made by Him should be brought to nothing through the deceit wrought       upon man by the devil; and it was supremely unfitting that the work of God in       mankind should disappear, either through their own negligence or through the       deceit of evil spirits. As, then, the creatures whom He had created reasonable,       like the Word, were in fact perishing, and such noble works were on the road to       ruin, what then was God, being Good, to do? Was He to let corruption and death       have their way with them? In that case, what was the use of having made them in       the beginning? Surely it would have been better never to have been created at       all than, having been created, to be neglected and perish; and, besides that,       such indifference to the ruin of His own work before His very eyes would argue       not goodness in God but limitation, and that far more than if He had never       created men at all. It was impossible, therefore, that God should leave man to       be carried off by corruption, because it would be unfitting and unworthy of       Himself.   

\textsc{(7)} Yet, true though this is, it is not the whole       matter. As we have already       noted, it was unthinkable that God, the Father of       Truth, should go back upon His word regarding death in order to ensure our       continued existence. He could not falsify Himself; what, then, was God to do?       Was He to demand repentance from men for their transgression? You might say that       that was worthy of God, and argue further that, as through the Transgression       they became subject to corruption, so through repentance they might return to       incorruption again. But repentance would not guard the Divine consistency, for,       if death did not hold dominion over men, God would still remain untrue. Nor does       repentance recall men from what is according to their nature; all that it does       is to make them cease from sinning. Had it been a case of a trespass only, and       not of a subsequent corruption, repentance would have been well enough; but when       once transgression had begun men came under the power of the corruption proper       to their nature and were bereft of the grace which belonged to them as creatures       in the Image of God. No, repentance could not meet the case. What—or rather Who       was it that was needed for such grace and such recall as we required? Who, save       the Word of God Himself, Who also in the beginning had made all things out of       nothing? His part it was, and His alone, both to bring again the corruptible to       incorruption and to maintain for the Father His consistency of character with       all. For He alone, being Word of the Father and above all, was in consequence       both able to recreate all, and worthy to suffer on behalf of all and to be an       ambassador for all with the Father.   

\textsc{(8)} For this purpose, then, the incorporeal and       incorruptible and immaterial Word of God entered our world. In one sense,       indeed, He was not far from it before, for no part of creation had ever been       without Him Who, while ever abiding in union with the Father, yet fills all       things that are. But now He entered the world in a new way, stooping to our       level in His love and Self-revealing to us. He saw the reasonable race, the race       of men that, like Himself, expressed the Father's Mind, wasting out of       existence, and death reigning over all in corruption. He saw that corruption       held us all the closer, because it was the penalty for the Transgression; He       saw, too, how unthinkable it would be for the law to be repealed before it was       fulfilled. He saw how unseemly it was that the very things of which He Himself       was the Artificer should be disappearing. He saw how the surpassing wickedness       of men was mounting up against them; He saw also their universal liability to       death. All this He saw and, pitying our race, moved with compassion for our       limitation, unable to endure that death should have the mastery, rather than       that His creatures should perish and the work of His Father for us men come to       nought, He took to Himself a body, a human body even as our own. Nor did He will       merely to become embodied or merely to appear; had that been so, He could have       revealed His divine majesty in some other and better way. No, He took our body,       and not only so, but He took it directly from a spotless, stainless virgin,       without the agency of human father—a pure body, untainted by intercourse with       man. He, the Mighty One, the Artificer of all, Himself prepared this body in the       virgin as a temple for Himself, and took it for His very own, as the instrument       through which He was known and in which He dwelt. Thus, taking a body like our       own, because all our bodies were liable to the corruption of death, He       surrendered His body to       death instead of all, and offered it to the Father. This       He did out of sheer love for us, so that in His death all might die, and the law       of death thereby be abolished because, having fulfilled in His body that for       which it was appointed, it was thereafter voided of its power for men. This He       did that He might turn again to incorruption men who had turned back to       corruption, and make them alive through death by the appropriation of His body       and by the grace of His resurrection. Thus He would make death to disappear from       them as utterly as straw from fire.   

\textsc{(9)} The Word perceived that corruption could not be got       rid of otherwise than through death; yet He Himself, as the Word, being immortal       and the Father's Son, was such as could not die. For this reason, therefore, He       assumed a body capable of death, in order that it, through belonging to the Word       Who is above all, might become in dying a sufficient exchange for all, and,       itself remaining incorruptible through His indwelling, might thereafter put an       end to corruption for all others as well, by the grace of the resurrection. It       was by surrendering to death the body which He had taken, as an offering and       sacrifice free from every stain, that He forthwith abolished death for His human       brethren by the offering of the equivalent. For naturally, since the Word of God       was above all, when He offered His own temple and bodily instrument as a       substitute for the life of all, He fulfilled in death all that was required.       Naturally also, through this union of the immortal Son of God with our human       nature, all men were clothed with incorruption in the promise of the       resurrection. For the solidarity of mankind is such that, by virtue of the       Word's indwelling in a single human body, the corruption which goes with death       has lost its power over all. You know how it is when some great king enters a       large city and dwells in one of its houses; because of his dwelling in that       single house, the whole city is honored, and enemies and robbers cease to molest       it. Even so is it with the King of all; He has come into our country and dwelt       in one body amidst the many, and in consequence the designs of the enemy against       mankind have been foiled and the corruption of death, which formerly held them       in its power, has simply ceased to be. For the human race would have perished       utterly had not the Lord and Savior of all, the Son of God, come among us to put       an end to death.   

\textsc{(10)} This great work was, indeed, supremely worthy of       the goodness of God. A king who has founded a city, so far from neglecting it       when through the carelessness of the inhabitants it is attacked by robbers,       avenges it and saves it from destruction, having regard rather to his own honor       than to the people's neglect. Much more, then, the Word of the All-good Father       was not unmindful of the human race that He had called to be; but rather, by the       offering of His own body He abolished the death which they had incurred, and       corrected their neglect by His own teaching. Thus by His own power He restored       the whole nature of man. The Savior's own inspired disciples assure us of this.       We read in one place: "For the love of Christ constraineth us, because we       thus judge that, if One died on behalf of all, then all died, and He died for       all that we should no longer live unto ourselves, but unto Him who died       and       rose again from the dead, even our Lord Jesus Christ."       And again another says:       "But we behold Him Who hath been made a little lower than the       angels, even Jesus, because of the suffering of death crowned with glory and       honor, that by the grace of God He should taste of death on behalf of every man."       The same writer goes on to point out why it was necessary for       God the Word and none other to become Man:       "For it became Him, for Whom are all things and through       Whom are all things, in bringing many sons unto glory, to make the Author of       their salvation perfect through suffering."              He means that the rescue of mankind from corruption was the proper       part only of Him Who made them in the beginning. He points out also that the       Word assumed a human body, expressly in order that He might offer it in       sacrifice for other like bodies:       "Since then the children are sharers in flesh and       blood, He also Himself assumed the same, in order that through death He might       bring to nought Him that hath the power of death, that is to say, the Devil,       and might rescue those who all their lives were enslaved by the fear of       death."       For by the sacrifice of His own body He did two things: He put an       end to the law of death which barred our way; and He made a new beginning of       life for us, by giving us the hope of resurrection. By man death has gained its       power over men; by the Word made Man death has been destroyed and life raised up       anew. That is what Paul says, that true servant of Christ:       "For since by man came death, by man came also the       resurrection of the dead. Just as in Adam all die, even so in Christ shall all       be made alive,"       and so forth. Now, therefore, when we die we no longer do so as men       condemned to death, but as those who are even now in process of rising we await       the general resurrection of all, "which in its own times He shall       show,"       even God Who wrought it and bestowed it on us.   

This, then, is the first cause of the Savior's becoming Man. There are,       however, other things which show how wholly fitting is His blessed presence in       our midst; and these we must now go on to consider.   

\section*{Chapter 3}



\textsc{(11)} When God the Almighty was making mankind through       His own Word, He perceived that they, owing to the limitation of their nature,       could not of themselves have any knowledge of their Artificer, the Incorporeal       and Uncreated. He took pity on them, therefore, and did not leave them destitute       of the knowledge of Himself, lest their very existence should prove purposeless.       For of what use is existence to the creature if it cannot know its Maker? How       could men be reasonable beings if they had no knowledge of the Word and Reason       of the Father, through Whom they had received their being? They would be no       better than the beasts, had they no knowledge save of earthly things; and why       should God have made them at all, if He had not intended them to know Him? But,       in fact, the good God has given them a share in His own Image, that is, in our       Lord Jesus Christ, and has made even themselves after the same Image and       Likeness. Why? Simply in order that through this gift of Godlikeness in       themselves they may be able to perceive the Image Absolute, that is the Word       Himself, and through Him to apprehend the Father; which knowledge of their Maker       is for men the only really happy and blessed life.   

But, as we have already seen, men, foolish as they are, thought little of the       grace they had received, and turned away from God. They defiled their own soul       so completely that they not only lost their apprehension of God, but invented       for themselves other gods of various kinds. They fashioned idols for themselves       in place of the truth and reverenced things that are not, rather than God Who       is, as St. Paul says, "worshipping the creature rather than the       Creator."       Moreover, and much worse, they transferred the honor which is due to God to       material objects such as wood and stone, and also to man; and further       even than       that they went, as we said in our former book. Indeed, so impious were they that       they worshipped evil spirits as gods in satisfaction of their lusts. They       sacrificed brute beasts and immolated men, as the just due of these deities,       thereby bringing themselves more and more under their insane control. Magic arts       also were taught among them, oracles in sundry places led men astray, and the       cause of everything in human life was traced to the stars as though nothing       existed but that which could be seen. In a word, impiety and lawlessness were       everywhere, and neither God nor His Word was known. Yet He had not hidden       Himself from the sight of men nor given the knowledge of Himself in one way       only; but rather He had unfolded it in many forms and by many ways.   

\textsc{(12)} God knew the limitation of mankind, you see; and       though the grace of being made in His Image was sufficient to give them       knowledge of the Word and through Him of the Father, as a safeguard against       their neglect of this grace, He provided the works of creation also as means by       which the Maker might be known. Nor was this all. Man's neglect of the       indwelling grace tends ever to increase; and against this further frailty also       God made provision by giving them a law, and by sending prophets, men whom they       knew. Thus, if they were tardy in looking up to heaven, they might still gain       knowledge of their Maker from those close at hand; for men can learn directly       about higher things from other men. Three ways thus lay open to them, by which       they might obtain the knowledge of God. They could look up into the immensity of       heaven, and by pondering the harmony of creation come to know its Ruler, the       Word of the Father, Whose all-ruling providence makes known the Father to all.       Or, if this was beyond them, they could converse with holy men, and through them       learn to know God, the Artificer of all things, the Father of Christ, and to       recognize the worship of idols as the negation of the truth and full of all       impiety. Or else, in the third place, they could cease from lukewarmness and       lead a good life merely by knowing the law. For the law was not given only for       the Jews, nor was it solely for their sake that God sent the prophets, though it       was to the Jews that they were sent and by the Jews that they were persecuted.       The law and the prophets were a sacred school of the knowledge of God and the       conduct of the spiritual life for the whole world.   

So great, indeed, were the goodness and the love of God. Yet men, bowed down       by the pleasures of the moment and by the frauds and illusions of the evil       spirits, did not lift up their heads towards the truth. So burdened were they       with their wickednesses that they seemed rather to be brute beasts than       reasonable men, reflecting the very Likeness of the Word.   

\textsc{(13)} What was God to do in face of this dehumanising of       mankind, this universal hiding of the knowledge of Himself by the wiles of evil       spirits? Was He to keep silence before so great a wrong and let men go on being       thus deceived and kept in ignorance of Himself? If so, what was the use of       having made them in His own Image originally? It would surely have been better       for them always to have been brutes, rather than to revert to that condition       when once they had shared the nature of the Word. Again, things being as they       were, what was the use of their ever having had the knowledge of God? Surely it       would have been better for God never to have bestowed it, than that men should       subsequently be       found unworthy to receive it. Similarly, what possible profit       could it be to God Himself, Who made men, if when made they did not worship Him,       but regarded others as their makers? This would be tantamount to His having made       them for others and not for Himself. Even an earthly king, though he is only a       man, does not allow lands that he has colonized to pass into other hands or to       desert to other rulers, but sends letters and friends and even visits them       himself to recall them to their allegiance, rather than allow His work to be       undone. How much more, then, will God be patient and painstaking with His       creatures, that they be not led astray from Him to the service of those that are       not, and that all the more because such error means for them sheer ruin, and       because it is not right that those who had once shared His Image should be       destroyed.   

What, then, was God to do? What else could He possibly do, being God, but       renew His Image in mankind, so that through it men might once more come to know       Him? And how could this be done save by the coming of the very Image Himself,       our Savior Jesus Christ? Men could not have done it, for they are only made       after the Image; nor could angels have done it, for they are not the images of       God. The Word of God came in His own Person, because it was He alone, the Image       of the Father Who could recreate man made after the Image.   

In order to effect this re-creation, however, He had first to do away with       death and corruption. Therefore He assumed a human body, in order that in it       death might once for all be destroyed, and that men might be renewed according       to the Image. The Image of the Father only was sufficient for this need. Here is       an illustration to prove it.   

\textsc{(14)} You know what happens when a portrait that has       been painted on a panel becomes obliterated through external stains. The artist       does not throw away the panel, but the subject of the portrait has to come and       sit for it again, and then the likeness is re-drawn on the same material. Even       so was it with the All-holy Son of God. He, the Image of the Father, came and       dwelt in our midst, in order that He might renew mankind made after Himself, and       seek out His lost sheep, even as He says in the Gospel:       "I came to seek and to save that which was lost.       This also explains His saying to the Jews: ""       a He was not referring to a man's natural birth from his mother, as they thought,       but to the re-birth and re-creation of the soul in the Image of God.   

Nor was this the only thing which only the Word could do. When the madness of       idolatry and irreligion filled the world and the knowledge of God was hidden,       whose part was it to teach the world about the Father? Man's, would you say? But       men cannot run everywhere over the world, nor would their words carry sufficient       weight if they did, nor would they be, unaided, a match for the evil spirits.       Moreover, since even the best of men were confused and blinded by evil, how       could they convert the souls and minds of others? You cannot put straight in       others what is warped in yourself. Perhaps you will say, then, that creation was       enough to teach men about the Father. But if that had       been so, such great evils       would never have occurred. Creation was there all the time, but it did not       prevent men from wallowing in error. Once more, then, it was the Word of God,       Who sees all that is in man and moves all things in creation, Who alone could       meet the needs of the situation. It was His part and His alone, Whose ordering       of the universe reveals the Father, to renew the same teaching. But how was He       to do it? By the same means as before, perhaps you will say, that is, through       the works of creation. But this was proven insufficient. Men had neglected to       consider the heavens before, and now they were looking in the opposite       direction. Wherefore, in all naturalness and fitness, desiring to do good to       men, as Man He dwells, taking to Himself a body like the rest; and through His       actions done in that body, as it were on their own level, He teaches those who       would not learn by other means to know Himself, the Word of God, and through Him       the Father.   

\textsc{(15)} He deals with them as a good teacher with his       pupils, coming down to their level and using simple means. St. Paul says as       much:       "Because in the wisdom of God the world in its wisdom       knew not God, God thought fit through the simplicity of the News proclaimed to       save those who believe."       Men had turned from the contemplation of God above, and were       looking for Him in the opposite direction, down among created things and things       of sense. The Savior of us all, the Word of God, in His great love took to       Himself a body and moved as Man among men, meeting their senses, so to speak,       half way. He became Himself an object for the senses, so that those who were       seeking God in sensible things might apprehend the Father through the works       which He, the Word of God, did in the body. Human and human minded as men were,       therefore, to whichever side they looked in the sensible world they found       themselves taught the truth. Were they awe-stricken by creation? They beheld it       confessing Christ as Lord. Did their minds tend to regard men as Gods? The       uniqueness of the Savior's works marked Him, alone of men, as Son of God. Were       they drawn to evil spirits? They saw them driven out by the Lord and learned       that the Word of God alone was God and that the evil spirits were not gods at       all. Were they inclined to hero-worship and the cult of the dead? Then the fact       that the Savior had risen from the dead showed them how false these other       deities were, and that the Word of the Father is the one true Lord, the Lord       even of death. For this reason was He both born and manifested as Man, for this       He died and rose, in order that, eclipsing by His works all other human deeds,       He might recall men from all the paths of error to know the Father. As He says       Himself,       "I came to seek and to save that which was lost."

\textsc{(16)} When, then, the minds of men had fallen finally to       the level of sensible things, the Word submitted to appear in a body, in order       that He, as Man, might center their senses on Himself, and convince them through       His human acts that He Himself is not man only but also God, the Word and Wisdom of       the true God. This is what Paul wants to tell us when he says:       "That ye, being rooted and grounded in love, may be       strong to apprehend with all the saints what is the length and breadth and       height and depth, and to know the love of God that surpasses knowledge, so       that ye may be filled unto all the fullness of God."        The Self-revealing of the Word is in every dimension—above, in       creation; below, in the Incarnation; in the depth, in Hades; in the breadth,       throughout the world. All things have been filled with the knowledge of God.   

For this reason He did not offer the sacrifice on behalf of all immediately       He came, for if He had surrendered His body to death and then raised it again at       once He would have ceased to be an object of our senses. Instead of that, He       stayed in His body and let Himself be seen in it, doing acts and giving signs       which showed Him to be not only man, but also God the Word. There were thus two       things which the Savior did for us by becoming Man. He banished death from us       and made us anew; and, invisible and imperceptible as in Himself He is, He       became visible through His works and revealed Himself as the Word of the Father,       the Ruler and King of the whole creation.   

\textsc{(17)} There is a paradox in this last statement which we       must now examine. The Word was not hedged in by His body, nor did His presence       in the body prevent His being present elsewhere as well. When He moved His body       He did not cease also to direct the universe by His Mind and might. No. The       marvelous truth is, that being the Word, so far from being Himself contained by       anything, He actually contained all things Himself. In creation He is present       everywhere, yet is distinct in being from it; ordering, directing, giving life       to all, containing all, yet is He Himself the Uncontained, existing solely in       His Father. As with the whole, so also is it with the part. Existing in a human       body, to which He Himself gives life, He is still Source of life to all the       universe, present in every part of it, yet outside the whole; and He is revealed       both through the works of His body and through His activity in the world. It is,       indeed, the function of soul to  things that are outside the body, but it       cannot energize or move them. A man cannot transport things from one place to       another, for instance, merely by thinking about them; nor can you or I move the       sun and the stars just by sitting at home and looking at them. With the Word of       God in His human nature, however, it was otherwise. His body was for Him not a       limitation, but an instrument, so that He was both in it and in all things, and       outside all things, resting in the Father alone. At one and the same time—this       is the wonder—as Man He was living a human life, and as Word He was sustaining       the life of the universe, and as Son He was in constant union with the Father.       Not even His birth from a virgin, therefore, changed Him in any way, nor was He       defiled by being in the body. Rather, He sanctified the body by being in it. For       His being in everything does not mean that He shares the nature of everything,       only that He gives all things their being and sustains them in it. Just as the       sun is not defiled by the contact of its rays with earthly objects, but rather       enlightens and purifies them, so He Who made the sun is not defiled by being       made known in       a body, but rather the body is cleansed and quickened by His       indwelling,       "Who did no sin, neither was guile found in His       mouth."

\textsc{(18)} You must understand, therefore, that when writers       on this sacred theme speak of Him as eating and drinking and being born, they       mean that the body, as a body, was born and sustained with the food proper to       its nature; while God the Word, Who was united with it, was at the same time       ordering the universe and revealing Himself through His bodily acts as not man       only but God. Those acts are rightly said to be His acts, because the body which       did them did indeed belong to Him and none other; moreover, it was right that       they should be thus attributed to Him as Man, in order to show that His body was       a real one and not merely an appearance. From such ordinary acts as being born       and taking food, He was recognized as being actually present in the body; but by       the extraordinary acts which He did through the body He proved Himself to be the       Son of God. That is the meaning of His words to the unbelieving Jews:       "If I do not the works of My Father, believe Me not;       but if I do, even if ye believe not Me, believe My works, that ye may know       that the Father is in Me and I in the Father."

Invisible in Himself, He is known from the works of creation; so also, when       His Godhead is veiled in human nature, His bodily acts still declare Him to be       not man only, but the Power and Word of God. To speak authoritatively to evil       spirits, for instance, and to drive them out, is not human but divine; and who       could see-Him curing all the diseases to which mankind is prone, and still deem       Him mere man and not also God? He cleansed lepers, He made the lame to walk, He       opened the ears of the deaf and the eyes of the blind, there was no sickness or       weakness that He did not drive away. Even the most casual observer can see that       these were acts of God. The healing of the man born blind, for instance, who but       the Father and Artificer of man, the Controller of his whole being, could thus       have restored the faculty denied at birth? He Who did thus must surely be       Himself the Lord of birth. This is proved also at the outset of His becoming       Man. He formed His own body from the virgin; and that is no small proof of His       Godhead, since He Who made that was the Maker of all else. And would not anyone       infer from the fact of that body being begotten of a virgin only, without human       father, that He Who appeared in it was also the Maker and Lord of all beside?   

Again, consider the miracle at Cana. Would not anyone who saw the substance       of water transmuted into wine understand that He Who did it was the Lord and       Maker of the water that He changed? It was for the same reason that He walked on       the sea as on dry land—to prove to the onlookers that He had mastery over all.       And the feeding of the multitude, when He made little into much, so that from       five loaves five thousand mouths were filled—did not that prove Him none other       than the very Lord Whose Mind is over all?   

\end{document}
